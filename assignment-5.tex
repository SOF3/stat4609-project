% This version of CVPR template is provided by Ming-Ming Cheng.
% Please leave an issue if you found a bug:
% https://github.com/MCG-NKU/CVPR_Template.

\documentclass[final]{cvpr}

\usepackage{times}
\usepackage{epsfig}
\usepackage{graphicx}
\usepackage{amsmath}
\usepackage{amssymb}

% Include other packages here, before hyperref.
\usepackage{acro}
\usepackage{csquotes}

% If you comment hyperref and then uncomment it, you should delete
% egpaper.aux before re-running latex.  (Or just hit 'q' on the first latex
% run, let it finish, and you should be clear).
\usepackage[pagebackref=true,breaklinks=true,colorlinks,bookmarks=false]{hyperref}


\def\cvprPaperID{****} % *** Enter the CVPR Paper ID here
\def\confYear{2021}
%\setcounter{page}{4321} % For final version only

\newcommand{\q}[1]{\enquote{#1}}

\newcommand{\acdef}[2]{
	\DeclareAcronym{#1}{
		short = #1,
		long = #2,
	}
}
\acdef{RMSE}{root mean squared error}
\acdef{KNN}{$k$-nearest neighbours}
\acdef{SVD}{singular value decomposition}
\acdef{NCF}{neural collaborative filtering}
\acdef{CF}{collaborative filtering}
\acdef{SGD}{stochastic gradient descent}


\begin{document}

%%%%%%%%% TITLE
\title{
	Assignment 5 \\~\\
	\large{Team name: M2 Robo}
}

\author{
	Lee Chun Yin\\
	3035469140\\
	\and
	Chiu Yu Ying\\
	3035477630
	\and
	Chan Kwan Yin\\
	3035466978 \\
	Team leader
}

\maketitle

\clearpage

\section{Introduction}
\subsection{\ac{CF}}
\ac{CF} is a technique for recommendation system,
in which historical feedback data are used to infer connections between users and products~\cite{FactorMeet}.
While additional features can be introduced to offset certain bias effects~\cite{BellKor2008},
two inputs (user and product) and one output (user rating on the product) are generally sufficient
to train a \ac{CF} model without involving domain-specific data.

Two major approaches for \ac{CF} include neighbourhood models and latent factor models.
Neighbourhood models compare the similarity between users
and recommend products positively rated by similar users,
while latent factor models perform dimensional reduction on both users and movies
to a common, smaller set of feature attributes such that
users are recommended with movies of more coherent features.

\subsection{The Netflix Prize dataset}
The Netflix Prize is a competition for the prediction of users' favour of movies.
The dataset provides existing ratings of users on given movies,
and models are trained to predict new ratings.

\subsubsection{Dataset format}

The dataset contains 100480507 rows of data structured in the following format:

\begin{tabular}{|c|c|c|c|}
	\hline
	User ID & 490189 discrete values \\ \hline
	Movie ID & 17770 discrete values \\ \hline
	Rating & 1, 2, 3, 4, 5 \\ \hline
	Date & Dates from 1999 to 2005 \\ \hline
\end{tabular}

\subsubsection{Distribution of ratings}
Ratings are mostly distributed around 3 and 4,
as shown in Figure \ref{fig:rating-freq}.

Except for some extreme cases, the number of ratings per user over the 7 years
mostly follow an exponential relation for users from 10 to 1000 ratings,
as shown in Figure \ref{fig:user-rating-freq}.
The top 10 users with the highest number of ratings
range from 17651 to 8877.

For movies with at least 100 ratings (which is the case for the majority),
their numbers of ratings demonstrate a similar but more concave relationship,
as shown in Figure \ref{fig:movie-rating-freq}.

\begin{figure}
	\includegraphics[width=0.5\textwidth]{screenshot20210422222105.png}
	\caption{Frequency of ratings}
	\label{fig:rating-freq}
\end{figure}

\begin{figure}
	\includegraphics[width=0.5\textwidth]{screenshot20210422222340.png}
	\caption{Number of ratings per user}
	\label{fig:user-rating-freq}
\end{figure}

\begin{figure}
	\includegraphics[width=0.5\textwidth]{screenshot20210422223229.png}
	\caption{Number of ratings per movie}
	\label{fig:movie-rating-freq}
\end{figure}

\subsubsection{Evaluation process}
To evaluate performance, 1425333 rows (about 1.42\% of all data) are speified as the standard "probe".
We perform analysis in the following procedure:

\begin{enumerate}
	\item Train the model with the 99055174 non-probe rows.
	\item Predict user ratings with the 1425333 probe rows.
	\item Compute the \ac{RMSE} between the predicted data and actual data.
\end{enumerate}

In this project, we evaluate three models, namely:

\begin{itemize}
	\item \ac{KNN}, a simple neighbourhood model
	\item \ac{SVD}, a latent factor model that accounts for user bias
	\item \ac{NCF}, a latent factor model that represents features with neural network weights
\end{itemize}

Originally, we proposed three models KNN, SVD and SVD++ (a SVD model with time features).
Due to the high similarity between SVD and SVD++,
we proposed the use of NCF instead,
while the additional time effect bias in SVD++ is left out as a to-do.

\section{Conclusion}

\subsection{\ac{KNN} model}
\subsubsection{Methodology}
The original KNN model we proposed constructs an $m \times n$ matrix
(where $m, n$ are the numbers of users and movies) to store all ratings,
and an $m \times m$ matrix to store user-user similarity.
It occurred that this algorithm was too expensive in terms of memory usage,
and computation of $O(m^2)$ similarity was too time-consuming.
We instead adopted a technique called Top $Q$ optimization proposed by Hong et al~\cite{Alpher01},
which only selects neighbours from the $Q$ users with the most ratings.

Nevertheless, we are concerned that this would lead to unreliable results
due to potential spamming or otherwise "light-hearted" ratings.
As observed from Figure~\ref{fig:user-rating-freq},
the topmost user has 17651 ratings, implying that
he/she has on average rated almost 7 movies per day.
We are doubtful about the integrity of such data,
whether they are genuinely representative of the rest of the users.
This is also vulnerable to attacks from malicious robots
who attempt to hijack recommendation systems through automatic rating.
Even if each rating requires completion of a captcha,
it only requires 800 manual completions per day to completely fill the top $100$ users.
This optimization is also unable to cater for interests from minority groups.

\subsubsection{Findings}
We discovered that the result using cosine with mean-corrected ratings has the best performance among all (RMSE = 1.1046)
while the result using cosine with 3-corrected ratings does not work (with RMSE = 1.1800).
The Netflix rating is a 5-star rating which is like a Likert scale commonly used in research surveys.
In fact, such a scale has a known issue called the central tendency bias,
which means people having surveys tend to have a neutral or middle rating and avoid the endpoints of scale~\cite{stevens1971issues}.
The poor performance of 3-corrected rating shows the dataset does not exactly follow the same situation of the bias mentioned.

However, according to the frequency graph in \ref{fig:rating-freq}, we can see the ratings accumulate at 3-4,.
This may be because viewers, who are potentially giving unfavour rating,
will simply close the movie and choose the other one after clicking and watching the preview of certain movie.
The potential raters only remain those who are still interested in the movie after watching the preview.
With the central tendency bias, those remaining viewers have a high tendency to give a rating among the favorable rating range (i.e. between 3 and 5).
In this case, 4 is the “middle” to be selected.
Apart from that, we also observe that there are fewer unfavorable (less than 3) ratings.
This may be because only those viewers, who watched most of the movie, feel disappointed will give unfavorable ratings.
This may be the reason for the less counts for unfavorable ratings.

Therefore, based on these cases, the “middle” of the rating is not just 3 and also 4.
The mean-corrected ratings can consider the mentioned situation.

\subsection{\ac{SVD} model}
\subsubsection{Methodology}


\subsubsection{Findings}
In the SVD model, we observed that train and test RMSE are inversely related
over different selections of hyperparameters.
Figure \ref{fig:svd-rmse-scatter} visualizes the results of different hyperparameters we tested with.
This suggests that the issue of overfitting is very prevalent in latent factor models.

In particular, the two models with the lowest test RMSE reduced the factorization dimension greatly
down to $k=50$ and $k=10$ respectively.
This suggests that the model starts memorizing specific users
once the the number of users is comparable to the number of dimensions.
Regularization was similarly observed to have the effect of sacrificing train RMSE for test RMSE.

\begin{figure}
	\includegraphics[width=0.5\textwidth]{screenshot20210422225123.png}
	\caption{Train RMSE vs Test RMSE}
	\label{fig:svd-rmse-scatter}
\end{figure}

\subsection{\ac{NCF} model}
\subsubsection{Methdology}
For the neural network model, we mainly faced difficulties in the training process.
Firstly, due to the complexity of the neural network model,
we faced difficulties in the training process for the whole dataset,
and had to use optimization techniques such as increasing batch size in order to complete the training in reasonable time.

Second, for the matrix factorization part of the neural network model,
issues were faced for the training loss,
where the training loss was unable to converge to zero even without model regularization and was stuck at RMSE $\ge 1$.
More investigation on the model training and neural network design needs to be done
to make sure the network is designed such that the loss can converge to a low value.

{\small
	\bibliographystyle{ieee_fullname}
	\bibliography{egbib}
}

\end{document}
